\documentclass[11pt,leqno]{article}

\usepackage[utf8]{inputenc}
\usepackage{polski}
\usepackage{a4wide}
\usepackage{graphicx}
\usepackage{fancyhdr}

\usepackage{amsmath,amssymb}
\usepackage{bbm}
\usepackage{amsthm}

\setlength{\headheight}{14pt}

\pagestyle{fancy}
\lhead{UniLANChat}
\rhead{Podręcznik programisty}

\usepackage[
pdfborder={0,0,0},
pdftitle={UniLANChat -- podręcznik programisty},
pdfauthor={Tomasz Wasilczyk},
pdfdisplaydoctitle=true
]{hyperref}

\def\changemargin#1{\list{}{\rightmargin#1\leftmargin#1}\item[]}
\let\endchangemargin=\endlist 

\makeatletter
	\renewcommand\@seccntformat[1]{\csname the#1\endcsname.\quad}
	\renewcommand\numberline[1]{#1.\hskip0.7em}
	\newcommand\BSLASH{\char`\\}
\makeatother

\author{Tomasz Wasilczyk, Piotr Gajowiak}
\title{UniLANChat -- podręcznik programisty}

\begin{document}

\pagenumbering{alph}
\begin{titlepage}
\begin{center}
	\vspace*{6cm}
	\textsc{\LARGE UniLANChat}\\[0.25cm]
	\textsc{\Large Podręcznik programisty}\\[1cm]
	Tomasz Wasilczyk, Piotr Gajowiak
	\vfill
	\today
\end{center}
\end{titlepage}
\pagenumbering{arabic}

\tableofcontents

\section{Wstęp}
UniLANChat to (docelowo) wieloprotokołowy, wieloplatformowy i wielojęzykowy klient czatów p2p.
Program powstał w celu zastąpienia komunikatora IP Messenger, używanego między innymi w akademikach
Uniwersytetu Wrocławskiego. Podstawowe założenia:
\begin{itemize}
	\item zgodność z protokołem IP Messenger, komunikatory powinny ze sobą współdziałać w ramach tej
	samej sieci;
	\item program powinien dobrze działać zarówno pod systemem Windows, jak i Linux -- napisany jest
	więc w JavaSE;
	\item ergonomiczny interfejs użytkownika, wzorowany jest na komunikatorze Pidgin;
	\item udostępniony na otwartej licencji LGPL;
	\item obsługa wielu języków (w przyszłych wersjach).
\end{itemize}

Pierwowzór odbiegał od części z powyższych założeń -- interfejs jest w nim jeszcze z czasów
Windows 3.11, a jego działanie pod systemem Linux pozostawia wiele do życzenia.

\vspace{1cm}
W miarę możliwości czasowych, być może będą realizowane następujące cele dodatkowe:
\begin{itemize}
	\item zgodność z innymi protokołami, np. LanChat, winpopup;
	\item różne interfejsy użytkownika: graficzny, tekstowy dla konsoli (czyli okienka ASCII) oraz
	tekstowy dla terminala (np. do obsługi standardowymi strumieniami);
	\item protokół automatycznego routingu: sieć p2p sama mogła by dbać, aby różni klienci,
	którzy nie posiadają bezpośredniego połączenia, mogli ze sobą rozmawiać (per proxy, przez innych
	użytkowników sieci);
	\item komunikacja głosowa;
	\item całkowicie zdecentralizowana sieć wymiany plików p2p.
\end{itemize}

\subsection{Autorzy}

Autorzy programu:
\begin{itemize}
	\item Tomasz Wasilczyk (\href{http://www.wasilczyk.pl}{www.wasilczyk.pl}),
	\item Piotr Gajowiak.
\end{itemize}

Strona projektu: \href{http://unilanchat.googlecode.com}{unilanchat.googlecode.com}

Pierwsze wersje programu powstały w ramach projektu licencjackiego, prowadzonego na Instytucie
Informatyki Uniwersytetu Wrocławskiego (\href{http://www.ii.uni.wroc.pl}{www.ii.uni.wroc.pl}).

%%%%%%%%%%%%%%%%%%%%%%%%%%%%%%%%%%%%%%%%%%%%%%%%%%%%%%%%%%%%%%%%%%%%%%%%%%%%%%%%

\section{Ogólna dokumentacja programistyczna}

Dokumentacja kodu źródłowego znajduje się w oddzielnym dokumencie, wygenerowanym narzędziem javadoc.

\subsection{Numeracja wersji}

Wersje programu są numerowane według schematu:
\[
<major>.<minor>[.<release>][ (nightly)]
\]
gdzie:
\begin{itemize}
	\item \textbf{major} -- główny numer wersji programu, związany z bardzo znaczącymi zmianami.
	Zaczyna się od 0 (wersja testowa), następne są wersjami przeznaczonymi dla użytkowników
	końcowych;
	\item \textbf{minor} -- kolejne wydania programu, wprowadzają nowe funkcje, lub znaczne
	poprawki błędów. Numeracja może być wielocyfrowa, zaczyna się od liczby
	1 w przypadku gałęzi 0.x oraz od 0 w przypadku kolejnych;
	\item \textbf{release} -- drobne poprawki, które nie zostały uwzględnione w wydaniu minor,
	ale wymagające szybkiej naprawy, np. poprawki bezpieczeństwa, lub błędów
	uniemożliwiających korzystanie z programu. Wersja 0 nie jest dodawana
	do ciągu oznaczającego pełny numer wersji;
	\item \textbf{nightly} -- kompilacja zrzutu z svn. Jest to wersja kandydująca do tej bez
	oznaczenia nightly, zazwyczaj do wydania wersji finalnej się znacznie
	zmienia. Nie jest przeznaczona dla użytkowników końcowych, ani betatesterów.
\end{itemize}

\subsection{Zasady formatowania}
\begin{itemize}
	\item wcięcia tabulacjami;
	\item pusta linia na końcu pliku, bez białych znaków na końcach linii (również pustych);
	\item klamry otwierające i zamykające w nowej linii, w tej samej kolumnie co blok
	nadrzędny, blok podrzędny wcięty o jedną tabulację;
	\item komentarze javadoc:
	\begin{itemize}
		\item opis metody z dużej litery, zakończony kropką;
		\item opisy parametrów i zwracanych wartości z małej litery, bez kropki na końcu;
		\item deklaracje zwracanych wartości boolean:
		\texttt{@return <code>true</code>, jeżeli (...)};
	\end{itemize}
	\item importowane paczki w dwóch oddzielonych pustą linią sekcjach: java[x] oraz reszta.
\end{itemize}

\subsection{Struktura aplikacji}

UniLANChat składa się z następujących komponentów:
\begin{itemize}
	\item program działający w maszynie wirtualnej Javy, którego kod źródłowy znajduje się
	w katalogu \texttt{java};
	\item biblioteki JNI, wykorzystywane przez powyższy program, których kod źródłowy jest
	w katalogu \texttt{jni}. Aby uprościć proces kompilacji projektu (oraz zmniejszyć wymagane
	zależności), te biblioteki są dostarczane w postaci binarnej (dla wszystkich obsługiwanych
	platform) z możliwością ich rekompilacji;
	\item natywnych programów startowych (znajdujących się w katalogu \texttt{launcher}) dla
	systemów Windows oraz Linux. Pierwszy z nich jest generowany programem
	launch4j (\href{http://launch4j.sourceforge.net}{launch4j.sourceforge.net}), drugi jest skryptem w bashu.
\end{itemize}

\subsection{Przepływ danych}

W programie wykorzystano wzorzec MVC, w związku z czym\footnote{w Internecie krąży wiele
sprzecznych interpretacji tego wzorca -- aby uniknąć nieporozumień, zostanie to tutaj uściślone}:
\begin{itemize}
	\item \textbf{model} przechowuje pełne dane na temat implementowanych rozwiązań, np. treść
	wiadomości, jej autora, datę odebrania. Nie ma jednak możliwości bezpośredniego oddziaływania
	na widok, ani kontroler (chyba, że tamte go obserwują);
	\item \textbf{widok} służy do wyświetlania danych z modelu oraz wywoływania odpowiednich metod
	kontrolera, w celu operowania na danych. Widok nie powinien modyfikować bezpośrednio modelu;
	\item \textbf{kontroler} odpowiada za wykonywanie akcji zleconych przez widok, a także
	przechowywanie wygenerowanych obiektów z modelu (czyli np. listy użytkowników, listy pokoi
	rozmów). Kontroler nie może bezpośrednio modyfikować widoku, ale może na niego oddziaływać
	przez wzorzez obserwatora.
\end{itemize}

Wzorzec obserwatora pozwala na przekazywanie informacji do obiektów, nad którymi obserwowany
obiekt nie ma kontroli. Np. model kontaktu (osoby na liście kontaktów) sam w sobie nie może
modyfikować widoku listy kontaktów, ale jest przez nią obserwowany\footnote{w rzeczywistości model kontaktu
jest obserwowany przez model listy kontaktów, a ta dopiero przez widok listy kontaktów}. Wtedy
w przypadku jego zmiany (np. zmiana statusu dostępności) wysyła o tym powiadomienie wszystkim
obserwatorom, a więc także wspomnianemu widokowi.

%%%%%%%%%%%%%%%%%%%%%%%%%%%%%%%%%%%%%%%%%%%%%%%%%%%%%%%%%%%%%%%%%%%%%%%%%%%%%%%%

\section{Protokół: IP Messenger}

Protokół jest bezpołączeniowy, opiera się o rozgłaszanie swojej obecności poprzez broadcast, za
pomocą pakietów UDP. Wykorzystuje opcjonalne szyfrowanie, transfer plików (oparty o TCP), pozwala na
użycie opcjonalnego serwera pośredniczącego.

Strona projektu: \href{http://ipmsg.org}{ipmsg.org}.

Domyślnie wykorzystywanym portem (zarówno TCP, jak i UDP) jest 2425 -- aby zachować możliwość
pełnej komunikacji, należy mieć w firewallu otworzone je także na połączenia przychodzące.

\subsection{Pakiety UDP}

Pakiety UDP są przesyłane w następującym formacie (jako czysty tekst):
\begin{verbatim}
	<wersja protokołu>:<numer pakietu>:<login nadawcy>:<host nadawcy>:
	<polecenie i flagi>:<dane>\0
\end{verbatim}
gdzie:
\begin{itemize}
	\item \textbf{wersja protokołu} -- ipmsg korzysta tylko z wartości \texttt{1}, a implementacja
	iptux -- \texttt{1\_iptux\_0};
	\item \textbf{numer pakietu} (w formacie dziesiętnym) -- unikalny (w skali każdego rozmówcy, z którym wymieniamy pakiety)
	identyfikator pakietu, dzięki któremu można realizować m.in. kontrolę dostarczania wiadomości;
	\item \textbf{login nadawcy} -- w oryginalnej implementacji jest to login zalogowanego
	użytkownika w systemie nadawcy. W implementacji UniLANChat jest to po prostu jego nick (nie
	chcemy zdradzać wspomnianej nazwy użytkownika);
	\item \textbf{host nadawcy} -- nazwa hostu nadawcy;
	\item \textbf{polecenie i flagi} (w formacie dziesiętnym) -- przechowuje zarówno identyfikator polecenia, jak i jego
	flagi. Aby się do nich odwołać, należy wykonać operacje bitową AND z wartościami:
	\begin{itemize}
		\item \texttt{0x000000FF} -- identyfikator polecenia (opisane w klasie \texttt{IpmsgPacket},
		w pakiecie \texttt{protocols.ipmsg});
		\item \texttt{0xFFFFFF00} -- flagi bitowe, wykorzystywane zależnie od polecenia (opisane j/w);
	\end{itemize}
	\item \textbf{dane} zależne od konkretnego identyfikatora polecenia (np. treść wiadomości,
	dane n/t klucza itp.), mogą zawierać kolejne separatory (dwukropki);
	\item \texttt{\BSLASH0} -- znak null.
\end{itemize}

\subsubsection{\texttt{NOOP} -- pusty pakiet}

Pusty pakiet, nie udało nam się odkryć jego zastosowania. Nie obsługuje żadnych flag, ani niczego
w sekcji danych.

\subsubsection{reszta pakietów}

TODO

\subsection{Nagłówki strumieni TCP}

TODO

\subsection{Scenariusze komunikacji}

TODO

\end{document}
