\documentclass[11pt,leqno]{article}

\usepackage[utf8]{inputenc}
\usepackage{polski}
\usepackage{a4wide}
\usepackage{graphicx}
\usepackage{fancyhdr}

\usepackage{amsmath,amssymb}
\usepackage{bbm}
\usepackage{amsthm}

\setlength{\headheight}{14pt}

\pagestyle{fancy}
\lhead{UniLANChat}
\rhead{Podręcznik użytkownika}

\usepackage[
pdfborder={0,0,0},
pdftitle={UniLANChat - podrecznik użytkownika},
pdfauthor={Tomasz Wasilczyk},
pdfdisplaydoctitle=true
]{hyperref}

\def\changemargin#1{\list{}{\rightmargin#1\leftmargin#1}\item[]}
\let\endchangemargin=\endlist 

\makeatletter
	\renewcommand\@seccntformat[1]{\csname the#1\endcsname.\quad}
	\renewcommand\numberline[1]{#1.\hskip0.7em}
	\newcommand\BSLASH{\char`\\}
\makeatother

\author{Tomasz Wasilczyk, Piotr Gajowiak}
\title{UniLANChat -- podręcznik użytkownika}

\begin{document}

\pagenumbering{alph}
\begin{titlepage}
\begin{center}
	\vspace*{6cm}
	\textsc{\LARGE UniLANChat}\\[0.25cm]
	\textsc{\Large Podręcznik użytkownika}\\[1cm]
	Tomasz Wasilczyk, Piotr Gajowiak
	\vfill
	\today
\end{center}
\end{titlepage}
\pagenumbering{arabic}

\tableofcontents

\section{Instalacja}

Instalacja programu polega na wypakowaniu archiwum do dowolnego katalogu, do którego użytkownik
końcowy ma prawa odczytu. Nie jest wymagane prawo zapisu, ponieważ program do przechowywania danych
stałych (np. konfiguracyjnych) używa katalogu domowego użytkownika, a do danych tymczasowych,
katalogu tymczasowego systemu.

Program wymaga do działania Java Runtime Environment, najlepiej wersję pobraną bezpośrednio ze
strony producenta. Nie jest gwarantowane, że na innych implementacjach (np. OpenJDK) program będzie 
działał poprawnie.

\section{Uruchomienie}

Pod systemem Linux należy uruchomić skrypt \texttt{./UniLANChat} (być może będzie trzeba nadać mu
uprawnienia do wykonywania), natomiast pod systemem Windows uruchomić program \texttt{UniLANChat.exe}.
Można też bezpośrednio uruchomić program, dwukrotnie klikając na archiwum \texttt{UniLANChat.jar},
lub poleceniem \texttt{java -jar ./UniLANChat.jar}.

Przed uruchomieniem programu \textbf{należy wyłączyć program IPMessenger} oraz inne programy
korzystające z tego protokołu. W przeciwnym wypadku, nie będzie można dołączyć do sieci.

\subsection{Parametry uruchomienia}\label{sec:parameters}

Program przy uruchomieniu może przyjmować parametry:
\begin{itemize}
	\item \textbf{-\--nick} \texttt{<nick>} -- nick, którym będzie się posługiwał użytkownik --
	wyświetlany na listach kontaktów innych użytkowników;
	\item \textbf{-\--user-dir} \texttt{<dir>} -- katalog, w którym mają być przechowywane dane
	konfiguracyjne użytkownika (zamiast katalogu domowego). Potrzebne m.in. przy przygotowywaniu
	przenośnej (np. do instalacji na dysku USB) wersji programu.
\end{itemize}

\section{Konfiguracja}

Konfiguracja jest przeprowadzana w oknie konfiguracji (pozycja ``Ustawienia'' w menu ``Program'').
Przyciski ze znakiem zapytania są odnośnikami do bardziej szczegółowych opisów poszczególnych funkcji.

Można ją też przeprowadzić ręcznie, poprzez edycję pliku \texttt{config.xml}, który znajduje
się w katalogu użytkownika, czyli \texttt{\~{ }/.UniLANChat} dla Linuxa, oraz
\texttt{C:}\textbackslash \texttt{Documents and Settings}\textbackslash
\texttt{Uzytkownik}\textbackslash \texttt{Dane aplikacji}\textbackslash
\texttt{UniLANChat} w przypadku Windowsa.

Należy pamiętać, aby w czasie edycji tego pliku, program był wyłączony. W przeciwnym razie być może
nadpisze on wprowadzone zmiany, bez ich odczytywania. Plik musi być zapisany w formatowaniu UTF-8.

Poza konfiguracją edytowalną z poziomu programu, w pliku znajduje się jedna dodatkowa opcja --
\textbf{debugMode}, zaznaczająca czy ma być włączony tryb debug.

\section{Automatyczna aktualizacja}

Program korzysta z modułu automatycznej aktualizacji. W tej chwili działa on w ten sposób, że
pobiera z serwera numer najnowszej wersji oraz porównuje z własną. Jeżeli się nie zgadzają,
wyświetla komunikat z powiadomieniem, po którego zaakceptowaniu użytkownik jest odsyłany na stronę
projektu, aby użytkownik ręcznie zainstalował sobie nową wersję.

\section{Interfejs użytkownika}

W tej chwili jedynym zaimplementowanym rodzajem interfejsu użytkownika jest interfejs okienkowy.
Jest on wzorowany na innych, popularnych komunikatorach, a przede wszystkim programie
Pidgin\footnote{http://www.pidgin.im}.

Jego głównym komponentem jest ikona w obszarze powiadomień systemu (Tray), za pośrednictwem której
użytkownik może uzyskać dostęp do okna głównego (listy plików) oraz odbierać powiadomienia o nowych
wiadomościach (jeżeli są jakieś nieprzeczytane wiadomości, kliknięcie w ikonę przeniesie do okna
rozmowy, zamiast do listy kontaktów).

\subsection{Lista kontaktów}

Okno listy kontaktów zawiera przede wszystkim samą listę z kontaktami (aby rozpocząć rozmowę
z wybranym kontaktem, należy dwa razy kliknąć na napis mu odpowiadający). Nad listą z kontaktami
znajduje się przycisk głównego pokoju rozmów, który służy do komunikacji ze wszystkimi dostępnymi
kontaktami. Na samej górze okna znajduje się menu, przez które jest dostęp do widoku transferowanych
plików oraz kończenia pracy programu.

Pod listą plików znajduje się panel wyboru statusu, na którym możemy wybrać jego rodzaj (czy jesteśmy
dostępni, ,,zajęci'', czy rozłączeni), a także status opisowy: aby go zmienić, należy wybrać
z menu opcję ,,Zmień opis'', wpisać opis w pole, które pojawi się pod spodem, a następnie
zatwierdzić enterem, albo przyciskiem po prawej. Status opisowy wyświetla się na liście kontaktów,
po prawej strony od użytkownika, który go sobie ustawił.

Elementy na liście kontaktów składają się z nazwy kontaktu, jego statusu (ikona po lewej), jego
statusu opisowego (szary tekst po prawej) oraz ikony protokołu (w tej chwili obsługiwany jest tylko
IPMsg).

\subsection{Lista przesyłanych plików}

Lista przesyłanych plików zawiera zarówno pliki wysyłane, jak i odbierane. Na uwagę zasługuje sposób
obliczania rozmiaru przesyłanych folderów. W przypadku komunikacji z drugim egzemplarzem programu
UniLANChat, jest to prawidłowy rozmiar całego przesyłanego katalogu. Oryginalna implementacja
IPMessengera nie dostarcza jednak takich informacji, więc rozmiar katalogu jest obliczany w trakcie
jego przesyłania. Skutkuje to efektem ,,skakania'' paska postępu. Gdy osiąga on maksimum, pasek
cofa się o pewien zakres, a następnie znowu postępuje -- działa to w ten sposób, ponieważ po
przesłaniu całości pliku program dopiero otrzymuje informacje o kolejnym i aktualizuje rozmiar
katalogu.

\subsection{Okno rozmów}

Okno rozmów zawiera zakładki z rozmowami, które można zamykać przez kliknięcie na wybraną środkowym
przyciskiem myszy, lub przez opcję z menu (Rozmowa - Zamknij rozmowę). W centralnej części okna
wyświetlony jest przebieg rozmowy, pod którym jest pole wpisywania wiadomości. Aby wprowadzić
wiadomość wielolinijkową, należy użyć kombinacji Shift-Enter. Aby wysłać wprowadzoną wiadomość należy
użyć przycisku Enter.

Aby dołączyć do wiadomości pliki, należy wybrać opcję z menu (Rozmowa - Wyślij plik), wybrać plik,
który następnie pojawi się nad polem wpisywania wiadomości. Po wpisaniu wiadomości należy ją
zatwierdzić klawiszem Enter.

Po prawej stronie od wiadomości na liście widoczne są klikalne ikony:
\begin{itemize}
	\item \textbf{ikona statusu wiadomości} -- informuje o postępie w dostarczaniu wiadomości
	do adresatów (być może wielu -- w pokojach publicznych):
	\begin{itemize}
		\item szara -- wiadomość nie posiada adresatów (np. jeżeli nikogo nie ma w pokoju publicznym);
		\item żółta -- wiadomość jest rozsyłana;
		\item zielona -- wiadomość udało się dostarczyć do wszystkich adresatów;
		\item czerwona -- nie udało się dostarczyć wiadomości do wszystkich adresatów (być może
		do kogoś się udało);
	\end{itemize}
	po kliknięciu na tą ikonę, pod wiadomością wyświetlają się szczegółowe listy (do kogo się udało
	dostarczyć);
	\item \textbf{ikona oryginalnej wiadomości} -- oryginalna implementacja IPMsg do każdej odpowiedzi
	dołącza wiadomość, na którą się odpowiada (podobnie, jak w komunikacji email). Przy prezentacji
	w formie rozmowy, utrudniało by to czytanie, więc wszystkie linie oznaczone jako cytat są
	przed wyświetleniem usuwane. Jeżeli jednak użytkownik zechce obejrzeć oryginalną wiadomość,
	wystarczy że kliknie na opisywaną ikonę z literą ,,R'';
	\item \textbf{ikona załącznika} -- jeżeli widnieje przy wysyłanej wiadomości, po jej kliknięciu
	pojawi się po prostu lista wysłanych plików. Jeżeli jednak to inny użytkownik przesyła pliki lub
	foldery, obok każdego z nich pojawi się opcja zapisu na dysk oraz (w przypadku plików) opcja
	otworzenia bez wybierania konkretnego folderu. Na przykład, jeżeli jest to obrazek, po kliknięciu
	,,otwórz'' zostanie on automatycznie pobrany do folderu tymczasowego oraz otworzony w domyślnym
	programie graficznym.
\end{itemize}

W publicznym pokoju rozmów, nazwy wypowiadających się użytkowników są linkami, po kliknięciu w które
otwiera się prywatny pokój rozmowy z danym użytkownikiem.


\end{document}
